\title{Assignment 0}
\author{
        Brinal Savsaviya, CSE\\
                180010030\\
            \and
        Dhruv Jain, EE\\
        180020006\\
        \and
        Shivam Chaturvedi, CSE\\
                180010032\\
}
\date{\today}

\documentclass[12pt]{article}

\begin{document}
\maketitle

\begin{abstract}
In this assignment, we simulate a scenario where an infiltrator from Attacking Country (AC) 
is trying to cross a border with randomly switched motion sensors. We try to show the relation between width of the border and probability of
sensor being switched with time taken to cross the border.
\end{abstract}

\section{Introduction}
\paragraph{Outline}
One country, called the defending country (DC),
wishes to defend its border against another country, called the attacking coun-
try (AC), whose aim is to send an infiltrator to cross the border and enter DC’s
land. DC decides to deploy a wireless sensor network along the border. If a sen-
sor detects an infiltration attempt, DC can then send its troops to counter the
infiltration with “fire and fury” (Trump style!). Quite obviously, the infiltrator
would like to enter DC’s land without triggering any sensors.
\section{Attempt at solution}\label{attempt}
We create multiple classes to assist with the simulation. \textit{Border} class has two members,
length and width. Length is fixed at 1000 units whereas we vary width, \textbf{W} was per requirement. \textit{Sensor} class
has a method which takes a parameter, the probability \textbf{P}, and returns a boolean \texttt{True} if
the random number generated, \textbf{X} is less than \textbf{P}, else it returns \texttt{False}. The \textit{infiltrator}
 class has a method which takes care of the motion of the infiltrator. If any 3 of the cells in front of him are empty, we moves forward,
 else we stays at his current cell. The method returns the sum of the width and the times he had to wait to move forward.
\textit{Time} class takes care of the time elapsed in this simulated world, and finally \textit{timeRequired}
 class is the main program which runs two loops for \textbf{P} from 0 to 0.99 \& \textbf{W} from 0 to 1000. It finally calculates
 the time taken for the infiltrator to cross the border and outputs it on the \textit{System Output}.

\section{Results}\label{results}
On looking at the data, we see that when we fix the width of the border, the time taken to cross the border looks linear till probability reaches really high values.
As soon as the probability becomes \(\geq 0.9\), the time taken starts shooting up exponentially.
\newline
**Insert graph here**
\newline
On the other hand, when we keep probability constant and vary width, the time taken to cross the border,
grows linearly with the width of the border.
\newline
**Insert graph here**

\section{Conclusions}\label{conclusions}
We can conclude that, when probability of the sensor being switched on becomes very large (\textit{when it is very likely that the sensor is on!}),
the infiltrator has to wait for a long time to cross even a single unit across the width. Whereas, for 
smaller probabilities, it is very unlikely that there is no path for the infiltrator to move forward.
Hence, it varies linerly with width.

\bibliographystyle{abbrv}
\bibliography{main}

\end{document}